\section{Design of Experiments}
When making experiments and performing analyses the experimenter makes decisions about which factors affect the experiment outcome and which are worthwhile to take into consideration when designing the experiment.
A direct approach to analysing the chosen factors is by changing one factor (switching its level) while keeping the others constant.
This is the best guess approach and can work well if the experimenter has intricate knowledge about the system.
However there a many disadvantages to this approach.
The One factor at a time approach consists of selecting baseline values for all factors and varying a single factor over all its levels while keeping all others constant.
This however can miss interaction effects.

The factorial approach to experiment design however allows one to take these effects into consideration by varying multiple factors together.
The number of experiments or runs required in the factorial design  follows the equation  $n^k$
where k is the number of factors you want to analyse and n is the number of levels, so if you want to change the flux and recovery of a system you have 2 factors and maybe you decide to have 3 values for each which would mean there would be 3 levels for each and the no. of experiments required would be $3^2=9$.
The factorial experiment gives more information at fewer experiments and therefore is generally more efficient than other experimental designs, especially as the number of factors increases.
This means that with increasing no of factors and levels the number of experiments required quickly rises to unpractical levels.
The fractional factorial approach allows you to cut the amount of required experiments by only performing a \textit{fraction} of the 'total' amount of experiments (as would be in the factorial approach).
The effect of a factor or the main effect is change in result based on a change in the level of the factor (for all results).
Some factors may have some degree of interaction between them.
If increasing a factor A results in a higher result at a low level of factor B but in a lower result at high B there is interaction between the two factors.


$3^3$ design %from brunos book
$3^3=27$ and has different factors at 3 different levels.
this experiment has 26 degrees of freedom. 


% \[
% y_{ijk}=\mu+\tau_i+\beta_j+(\tau\beta)_{ij}+\varepsilon_{ijk}
%     \left\{
%     \begin{array}{1l}
%       i=1,2,...,a\\
% j=1,2,...,b\\
% k=1,2,...,n
%     \end{array}
%     \right.
% \]
