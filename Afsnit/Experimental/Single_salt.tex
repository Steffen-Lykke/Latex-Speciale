\section{Experimental Method}
\label{Binary_ion_experiment_consideration}


%In order to gain knowledge regarding rejection of the NF membrane performance and operation of the lab-scale filtration set-up, a binary ion experiment was designed.  
The binary ion solutions were made using either  \ce{NaCl}, \ce{CaCl2}  or \ce{Na2OSiO2}, based on the problematic species  which control COC during CT operation being \ce{Cl-} and \ce{SiO2}. 
%based on the primary species present in water from a CT reservoir (\ce{Ca^{2+}},  \ce{Na+}, \ce{Cl-}, \ce{SO4^{2-}} and \ce{SiO2}).
The concentrations of the solutions was selected based on \ce{SiO2} and \ce{Cl-} content in authentic CT reservoir water, measured internally by Grundfos see \cref{Tab:single_salt_conc}. 
The batch filtration of these solutions was used to determine baselines for membrane performance and ion retention on the lab-sale filtration set-up.
Each solution was made by adding the respective salts to water treated by reverse osmosis (RO water), it was decided to use RO water instead of for e.g. tap-water, to achieve  control of the species present and thus the environment of \ce{Cl-} and \ce{SiO2} during filtration. 
The solutions were bubbled with air over night to establish equilibrium of the carbonate system, and the pH of the binary ion solutions were adjusted to pH 8.5. 
The batch filtrations ran for 4 hours with an initial feed volume of 6 L.
Samples of the retentate, feed and permeate streams were collected throughout the filtration and analysed to determine the rejection. 


\begin{table}[H]
\centering
\caption{Concentration and content of binary ion solutions: \ce{NaCl}, \ce{CaCl2}  and \ce{Na2OSiO2}. }
\label{tab:SingleSalt_composition_2}
\rowcolors{2}{gray!25}{white}
\begin{tabular}{r|cccc}
\rowcolor{gray!50}
\textbf{Ion Species}   &  \textbf{\ce{NaCl}} & \textbf{\ce{CaCl2}} &\textbf{ \ce{Na2OSiO2}} \\ \hline
\ce{Ca^{2+}} {[}mM{]}   &                & 3   &   \\
\ce{Na+} {[}mM{]}   & 3             &   & 2    \\
\ce{Cl-} {[}mM{]}   & 3             & 6  &     \\
\ce{SiO2} {[}mM{]} &               &    & 1     \\

\end{tabular}
\end{table}

\begin{table}[H]
\centering
\caption{Concentration and content of binary ion solutions: \ce{NaCl}, \ce{CaCl2}  and \ce{Na2OSiO2}.}
\rowcolors{2}{gray!25}{white}
	\begin{tabular}{cc}
	\rowcolor{gray!50}
    \textbf{Species} & \textbf{Concentration mM }  \\ \hline
     \ce{NaCl} & 3  \\
     \ce{CaCl2}  & 3 \\
     \ce{Na2OSiO2}  &  1\\
          	\end{tabular}
	\label{Tab:single_salt_conc}
\end{table}

% %batch. 
% The experiment were performed as batch filtrations.
% When operating in batch mode the retentate stream is constantly lead back into the feed tank, this leads to increased concentration of contaminants in the feed tank, but allows for high water recovery. 
% As a result of the batch filtration mode the concentration in the feed stream increase, and it was therefore possible to investigate the development of rejection of \ce{Cl-} and \ce{SiO2} at various concentrations.
% %Water recovery was selected based on the recovery of NF-pilot plant which is 92.95 \%.
% On the smaller lab-scale setup specific recovery values were difficult to achieve accurately and  water recovery aimed at being in the range of 90-95 \%.



% Both retentate and feed samples were collected to investigate the concentration profile across the length of the membrane. 
% %The samples were taken at 0h, 1h, 2h, 3h, 3.5h and 4h  and analysed by IC \rod{hvorfor vi bruger IC}. 




   






