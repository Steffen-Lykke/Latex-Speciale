
\subsection{Membrane materials and construction}

Membranes can be constructed in a variety of different configurations; flat-sheet (find kilde), hollow fiber (find kilde), spiral wound etc. 
Spiral bound is a membrane construction where the membrane sheets are stacked and rolled alternating with separation sheet creating a channel for the feed/retentate stream and permeate stream, separated by the membrane. \citep{keo}

NF membranes are typically made from polymers ( polysulphone, polyamide) or ceramic materials.
%Polysulfone membranes exhibit a wide pH tolerance (continuousexposure range of pH 1–13), high temperature limit (typically 75C), good oxidantresistance (chlorine exposure: storage 50 mg/l, short-term sanitation 200 mg/l,>106ppm h)
%The outside-in scheme has the advantage of creating larger membranesurface area, accepting solids of bigger size hence minimizing pretreatment needsand reducing pressure losses occurring inside thefibers. On the other hand, theinside-out configuration produces more efficient backwash and has the principaladvantage of working in crossflow mode in most systems which allows efficientflushing of the solids.
%^https://onlinelibrary-wiley-com.zorac.aub.aau.dk/doi/pdf/10.1002/9783527631407

\textcolor{magenta}{Noget med morphology(?) altså er det tynd film med support lag eller? porosity?}

\textcolor{blue}{NF bogen \citep{nanofiltration_2021_bog_fraMorten}}
Advantages of Hollow Fiber membranes: Surface to Volume ratio, Easy hydraulic cleaning ( backwash)


crossflow:

\begin{ceqn}
 \begin{align}
      u=\frac{Q/N}{\pi d_t^2/4}
    \end{align}
\end{ceqn}

 
 $K_w=Q_p/NAP*A$\\
 $K_s=Q_pc_p/A\Delta c$\\
 $\Delta c=c_m-c_p=\frac{c_f+c_c}{2}-c_p$\\
 $c_p=\frac{K_sc_f}{K_w\Delta P+\Delta \pi \frac{2-2R}{2R}+K_s}$

