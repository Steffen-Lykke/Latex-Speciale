\section{Binary ion Experiment}
\label{Binary_ion_experiment_consideration}



\rod{Mangler: noget med vi bobler med akvarie pumpen, har indstillet til pH 8,5 EFTER vi har boblet.}

%In order to gain knowledge regarding rejection of the NF membrane performance and operation of the lab-scale filtration set-up, a binary ion experiment was designed.  

As mentioned in \textcolor{blue}{afgrænsningen} the primary ions present in water from a CT reservoir are  \ce{Ca^{2+}},  \ce{Na+}, \ce{Cl-} and \ce{SO4^{2-}} as well as \ce{SiO2}. 
The problematic species theorized to control COC are mainly \ce{Cl-} and \ce{SiO2}. %(apart from \ce{Ca^{2+}}). 
\ce{Cl-} and \ce{SiO2} binary ion solutions were made using either  \ce{NaCl}, \ce{CaCl2}  or \ce{Na2OSiO2}, the concentrations of which are presented in \cref{Tab:single_salt_conc}. %as these salts are composed of the ions most abundant in CT reservoir water. %kilde
The concentrations of the solutions was selected based on \ce{SiO2} and \ce{Cl-} content in authentic CT reservoir water, measured internally by Grundfos (\textcolor{blue}{Vi kunne evt. smide dem i appendix}). 
The filtration of these solutions was used to determine baselines for membrane performance and ion retention on the lab-sale filtration set-up.
Each solution was made by adding the respective salts to water treated by reverse osmosis (RO water), it was decided to use RO water instead of for e.g. tap-water to have better control of the species present and thus the environment of \ce{Cl-} and \ce{SiO2} during filtration. 

\begin{table}[H]
\centering
\caption{Concentration and content of binary ion solutions, as well as \ce{Cl-}and \ce{SiO2} content.}
	\begin{tabular}{cc}
    Species & Concentration mM   \\ \midrule
     \ce{NaCl} & 3  \\
     \ce{CaCl2}  & 3 \\
     \ce{Na2OSiO2}  &  1\\
          	\end{tabular}
	\label{Tab:single_salt_conc}
\end{table}



%batch. 
The experiment were performed as batch filtrations.
When operating in batch mode the retentate stream is constantly lead back into the feed tank, this leads to increased concentration of contaminants in the feed tank, but allows for high water recovery. 
With the increase in concentration of the feed stream it was possible to investigate the development of rejection of \ce{Cl-} and \ce{SiO2} at various concentrations.
Water recovery was selected based on the recovery of NF-pilot plant which is 92.95 \%.
On the smaller lab-scale setup specific recovery values were difficult to achieve accurately and  water recovery aimed at being in the range of 90-95 \%.
The filtrations ran for 4 hours with an intial feed volume of 6 L.
Samples of the retentate, feed and permeate streams were analysed to determine the rejection. 
Both retentate and feed samples were collected to investigate the concentration profile across the length of the membrane. 
%The samples were taken at 0h, 1h, 2h, 3h, 3.5h and 4h  and analysed by IC \rod{hvorfor vi bruger IC}. 
The samples were analyzed for anions and cations using IC \textcolor{blue}{(stads)}, and bicarbonate was analyzed by a total alkalinity test \textcolor{blue}{(according to noget DS)}, at an internal Grundfos laboratory.                                     
Furthermore samples containing \ce{Na2OSiO2} were also analyzed for \ce{SiO2} using silicomolybdate method \textcolor{blue}{(performed with Hack lange test kits)}. 



   






