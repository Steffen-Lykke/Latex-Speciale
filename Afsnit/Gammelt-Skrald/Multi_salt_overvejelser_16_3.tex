\section{Complex ion matrix }
%\prettyinpink{Noget med water recovery}
In order to investigate the membrane performance for specific ions in a complex matrix an experiment similar to the binary ion experiments\rod{ref} was performed with more ions included in the matrix.
The same salts as in the single pass experiments\rod{ref til afgræsningen} were used to generate a matrix resembling authentic CT reservior water. 
The concentrations of each species are presented in \Cref{tab:multisalt_composition} and compared with \rod{the water quality previously determined to represent CT water}.
Furthermore the effect of solution pH was also investigated by adjusting the pH of the matrix to three different levels: pH 9.2 , pH 10, and pH 10.5. 
The pH values investigated are chosen to be in a range that would likely impact silica charge and thereby rejection \textcolor{blue}{ref til silica teori}.
%After these experiments the pH level can then further be adjusted to a more optimal value.
This best guess approach is not ideal from an experimental design point of view, but allows one to quickly gain information on how the system responds to the change in pH.
CT water has a buffer capacity which is not replicated when the simulated CT water exclusively consists of salts added to RO-water.
In order to mimic this buffering capacity of CT water the solutions were buffered using a carbonate-bicarboante buffer system.
A 0.008 M buffer was used as the base for all solutions and adjusted to the desired pH before filtration. 

The complex matrices were filtered in a batch process similar to filtration of the binary ions described in \textcolor{blue}{ref til afsnit med system beskrivelse}.
%This was chosen to examine how membrane performance (i.e. rejection of ions) changes as filtration progresses and ion concentrations increase in the feed stream.
The goal was to see the impact on Cl and silica rejection which should be able to be investigated.
The Cl rejection is expected to be dependent on the concentration of other species present in the feed tank / at the membrane wall \textcolor{blue}{ref til teori/ explanation} and therefor will likely change as the filtration progresses.
Silica rejection however, is expected to be less dependant on the matrix effects, but instead on the solution pH as this is responsible for the degree to which silica is charged.



\begin{table}[H]
\centering
\caption{Ionic Compositions of Matrices}
\label{tab:multisalt_composition}
\rowcolors{2}{gray!25}{white}
\begin{tabular}{l|cccc}
\rowcolor{gray!50}
\textbf{Ion Species}   & \textbf{Authentic CT water} & \textbf{pH 9.2} & \textbf{pH 10} &\textbf{ pH 10.5} \\ \hline
\ce{Na+} {[}mM{]}   & 10.81              & 11.77  & 14.76 & 16.63   \\
\ce{Cl-} {[}mM{]}   & 2.48               & 2.50   & 2.50  & 2.50    \\
\ce{SO_4^{2-}} {[}mM{]}  & 0.64               & 0.70   & 0.70  & 0.70    \\
\ce{Ca^{2+}} {[}mM{]}   & 0.35               & 0.40   & 0.40  & 0.40    \\
\ce{SiO2} {[}mM{]} & 0.68               & 0.70   & 0.70  & 0.70    \\
\ce{HCO_3^-}{[}mM{]}  & 7.29               & 6.07   & 3.08  & 1.21   
\end{tabular}
\end{table}

